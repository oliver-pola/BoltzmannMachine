\documentclass[
	12pt,
	a4paper,
	BCOR10mm,
	%chapterprefix,
	DIV14,
	headsepline,
	usegeometry,
	%twoside,
	%openright
]{scrreprt}

\KOMAoptions{
	listof=totoc,
	bibliography=totoc,
	index=totoc
}

\usepackage[T1]{fontenc}
\usepackage[utf8]{inputenc}

\usepackage{lmodern}

\usepackage[english]{babel}
%\usepackage[ngerman]{babel}

\usepackage[toc]{appendix}
\usepackage{color}
\usepackage{eurosym}
\usepackage{fancyhdr}
\usepackage{geometry}
\usepackage{graphicx}
\usepackage[htt]{hyphenat}
\usepackage{listings}
\usepackage{lstautogobble}
\usepackage{microtype}
\usepackage[list=true,hypcap=true]{subcaption}
\usepackage{textcomp}
\usepackage{units}
\usepackage{amsmath,amsthm}
\usepackage{amsfonts}
\usepackage{mathtools}
\usepackage{varioref}
\usepackage[hidelinks]{hyperref}
\usepackage[capitalise,noabbrev]{cleveref}
\usepackage{setspace}
\usepackage{tikz,pgfplots}
\usepackage{todonotes}
\usepackage[super]{nth}

\usetikzlibrary{positioning,calc}

\pgfplotsset{compat=newest}

\definecolor{uhhred}{cmyk}{0,1,1,0}
\definecolor{darkred}{cmyk}{0,1,1,0}

\lstset{
	frame=single,
	aboveskip=20pt,
	belowskip=10pt,
	basicstyle={\scriptsize\ttfamily},
	numbers=left,
	numberstyle=\tiny\color{gray},
	keywordstyle=\color{magenta},
	commentstyle=\color{gray},
	stringstyle=\color{teal},
	emphstyle=\color{blue},
	language=python,
	breaklines=true,
	breakatwhitespace=true,
	postbreak=\hbox{$\hookrightarrow$ },
	showstringspaces=false,
	autogobble=true,
	upquote=true,
	tabsize=4,
	captionpos=b,
	morekeywords={as,with}
}

% no pagebreaks for paper OUTLINE
\renewcommand\clearpage{}

\makeatletter
\renewcommand*{\lstlistlistingname}{List of Listings}
\addto{\captionsenglish}{%
  \renewcommand{\bibname}{References}
}
\makeatother

\begin{document}

\newgeometry{left=2cm, top=2cm, right=2cm, bottom=2cm}

\begin{titlepage}
	\includegraphics[width=0.5\textwidth]{fig/UHH-Logo_2010_Farbe_CMYK}

	\begin{center}
		{\Large \textcolor{uhhred}{\textbf{Seminar Paper}}\par}

		\vspace{1cm}

		{\titlefont\huge Boltzmann Machine\par}

		\vspace{1cm}

		{\large Neural Networks Seminar\par}

		\vspace{1cm}

		{\large Summer term 2020\par}

		\vspace{1cm}

		{\large by\par}

		\vspace{0.5cm}

		{\large Michael Blesel, Oliver Pola\par}
		
		% only for OUTLINE, DRAFT
		\vspace{2cm}
		{\huge\textcolor{uhhred}{\textbf{OUTLINE}}}
	\end{center}

	\vfill

	{\large\noindent\begin{tabular}{l}
		Knowledge Technology Research Group\\
		Department of Informatics\\
		Faculty of Mathematics, Informatics and Natural Sciences\\
		Universität Hamburg
	\end{tabular}\par}

	\vspace{1cm}

	{\large\noindent\begin{tabular}{ll}
		Supervisors:    		& Dr. Cornelius Weber\\
						& Dennis Becker\\
		\\
		Hamburg, xx.xx.2020
	\end{tabular}\par}
\end{titlepage}

\restoregeometry


\tableofcontents


\chapter*{Abstract}

Here will be an abstract.


\chapter{Introduction}

The idea of a Boltzmann Machine (BM) is quite old~\cite{Ackley}. Similar to neural networks in general, the topic has been asleep for quite a while. Now that we have the necessary computing power, it's time to revive those ideas.


\chapter{Boltzmann Machine}

Here we will give an introduction to the theoretical concept of a BM. It is based on concepts from physics, especially statistical mechanics, like the Ising model~\cite{Ising}. Idea is to have a stochastic neural network that finds a ``energy minimum'' while a overall ``temperature'' is slowly reduced.


\chapter{Related Work}

Here we will mention the works on BM we found and got inspiration from, but also outline what we will do different (see Variants).


\chapter{Implementation}

We will try to implement the concept of a BM in a modern framework like Tensorflow or PyTorch, if that makes sense. That will be experiments, where we can't predict the results yet.


\chapter{Variants}

When we have an implementation that can do what code of others can do as well, we will try to contribute some new approaches and maybe enlarge the scope of problems a BM can be used for.


\section{Time series data}

So far a BM was only applied to static data.


\section{Continuous data}

So far a BM was only applied to discrete data.


\chapter{Results}

We will try to find a suitable problem set, where we can apply our implementation and can compare it to other, more standard, solutions. Also, if applicable, we compare different variants / implementations of our BM.


\chapter{Discussion}

Do we need that? If it's rather short maybe included in the conclusion?  % TODO


\chapter{Limitations}

This might be optional. % TODO


\chapter{Conclusion}

We will conclude with our interpretation of the results and will point out where future work would be beneficial.


\bibliographystyle{apalike}
\bibliography{references}

\end{document}
